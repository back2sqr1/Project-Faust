\documentclass[11pt]{article}
\usepackage{latexsym}
\usepackage{amsmath}
\usepackage{amssymb}
\usepackage{amsthm}
\usepackage{epsfig}
\usepackage{graphicx}

\usepackage{amsmath}
\usepackage{amssymb}
\newcommand{\hw}[3]{
	\noindent
	\begin{center}
		\framebox{
			\vbox{
				\hbox to 5.78in { {\bf CSCE 653: Computational Methods for Data Science} \hfill  }
				\vspace{2mm}
				\hbox to 5.78in { {\Large \hfill Homework #1\hfill} }
				\vspace{2mm}
				\hbox to 5.78in { {\it Due date: #2 \hfil Name: #3} }
			}
		}
	\end{center}
	\vspace*{4mm}
}

%Problems, Answers and Solutions
\newcounter{prob}
\setcounter{prob}{-1}
\newcommand{\problem}{\stepcounter{prob}{\noindent\textbf{Problem \theprob.}}\ }
\newcommand{\answer}{\medskip{\color{red} \textbf{Answer to Problem \theprob.}}\ }
\newcommand{\solution}{\medskip{\color{blue} \textbf{Solution to Problem \theprob.}}\ }

%New commands
\def\ds{\displaystyle}
\def\ra{\rightarrow}
%\def\bf{\textbf}
\usepackage{enumerate}
\newcommand{\babc}{\begin{enumerate}[a)]} %Use with \item for abc lists
\newcommand{\eabc}{\end{enumerate}}


% For hyperlinking everything
\usepackage{hyperref}
\hypersetup{
	colorlinks=true, %set true if you want colored links
	linktoc=all,     %set to all if you want both sections and subsections linked
	linkcolor=blue,  %choose some color if you want links to stand out
}

\usepackage{tabularx}
\usepackage{booktabs}
\usepackage[latin1]{inputenc}
\usepackage{amsmath}
\usepackage{mathrsfs}  
\usepackage{amsfonts}
\usepackage{amssymb}
\usepackage{graphicx}
\usepackage{subfig}
\usepackage{caption}
\usepackage{algorithm}
%\usepackage{algcompatible}
%\usepackage{algorithmicx}
\usepackage{algpseudocode}
%\usepackage{enumitem}
\usepackage{color}
\usepackage{titlesec}
\titleformat{\section}{\fontfamily{lmss}\fontsize{14}{15}\bfseries}{\thesection}{1em}{}
\titleformat{\subsection}{\fontfamily{lmss}\fontsize{12}{15}\bfseries}{\thesubsection}{1em}{}

\usepackage{amsthm}

\newtheoremstyle{noit}
{10pt}% <Space above>
{10pt}% <Space below>
{}% <Body font>
{}% <Indent amount>
{\bfseries}% <Theorem head font>
{.}% <Punctuation after theorem head>
{.5em}% <Space after theorem headi>
{}% <Theorem head spec (can be left empty, meaning `normal')>

\newtheoremstyle{example}
{10pt}% <Space above>
{10pt}% <Space below>
{}% <Body font>
{20pt}% <Indent amount>
{\bfseries}% <Theorem head font>
{.}% <Punctuation after theorem head>
{.5em}% <Space after theorem headi>
{}% <Theorem head spec (can be left empty, meaning `normal')>


\newtheoremstyle{indented}{20pt}{20pt}{\addtolength{\leftskip}{2.5em}}{}{\bfseries}{.}{.5em}{}

%\theoremstyle{indented}
\newtheorem{claim}{Claim}
\newtheorem{theorem}{Theorem}
\numberwithin{theorem}{section}
\newtheorem{lemma}[theorem]{Lemma}
\newtheorem{corollary}[theorem]{Corollary}
\newtheorem{observation}{Observation}
%\numberwithin{observation}{section}
%\numberwithin{definition}{section}
\newtheorem{conjecture}{Conjecture}
\newtheorem{Qu}{Question}
\newcommand{\QU}{\begin{Qu}\normalfont}
	
	\newtheorem{definition}{Definition}
	
	\theoremstyle{noit}
	\newtheorem{fact}{Fact}
	
	%\theoremstyle{indented}
	
	
	\theoremstyle{indented}
	\newtheorem{example}{Example}
	
	\theoremstyle{indented}
	%\newtheorem{problem}{Problem}
	
	
	\newcommand{\vs}[1]{\vspace{#1}}
	
	\newcommand{\lecture}[1]{
		\noindent
		\begin{center}
			\framebox{
				\vbox{
					\hbox to 5.78in { {\bf CSCE 653: Computational Methods for Data Science} \hfill  }
					\vspace{2mm}
					\hbox to 5.78in { {\Large \hfill Lecture #1\hfill} }
					\vspace{2mm}
					\hbox to 5.78in { {Texas A\&M University \it  \hfill Lecturer: Nate Veldt} }
				}
			}
		\end{center}
		\vspace*{4mm}
	}
	
	
	\newcommand{\under}[1]{\underline{\hspace{#1}}}
	\setlength{\parindent}{0em}
	
	\newcommand\norm[1]{\left\lVert#1\right\rVert}
	\newcommand\normx[1]{\left\Vert#1\right\Vert}
	
	% Graph terms
	\newcommand{\vol}{\textbf{vol}}
	\newcommand{\cut}{\textbf{cut}}
	%\newcommand{\span}{\textbf{span}}
	
	% Matrices
	\newcommand{\mL}{\textbf{L}}
	\newcommand{\mR}{\textbf{R}}
	\newcommand{\mA}{\textbf{A}}
	\newcommand{\mB}{\textbf{B}}
	\newcommand{\mC}{\textbf{C}}
	\newcommand{\mM}{\textbf{M}}
	\newcommand{\mP}{\textbf{P}}
	\newcommand{\mZ}{\textbf{Z}}
	\newcommand{\mY}{\textbf{Y}}
	\newcommand{\mX}{\textbf{X}}
	\newcommand{\mI}{\textbf{I}}
	\newcommand{\mH}{\textbf{H}}
	\newcommand{\mW}{\textbf{W}}
	\newcommand{\mV}{\textbf{V}}
	\newcommand{\mU}{\textbf{U}}
	\newcommand{\mQ}{\textbf{Q}}
	% vectors
	
	\DeclareMathOperator*{\argmax}{argmax}
	\DeclareMathOperator*{\argmin}{argmin}
	
	\newcommand{\blue}[1]{{\color{blue} #1 }}
	\newcommand{\va}{\textbf{a}}
	\newcommand{\vb}{\textbf{b}}
	\newcommand{\vc}{\textbf{c}}
	\newcommand{\vd}{\textbf{d}}
	\newcommand{\ve}{\textbf{e}}
	\newcommand{\vh}{\textbf{h}}
	\newcommand{\vw}{\textbf{w}}
	\newcommand{\vl}{\textbf{l}}
	\newcommand{\vp}{\textbf{p}}
	\newcommand{\vq}{\textbf{q}}
	\newcommand{\vr}{\textbf{r}}
	\newcommand{\vu}{\textbf{u}}
	\newcommand{\vv}{\textbf{v}}
	\newcommand{\vx}{\textbf{x}}
	\newcommand{\vy}{\textbf{y}}
	\newcommand{\vz}{\textbf{z}}
	\newcommand{\zvec}{\textbf{0}}
	% Other
	\newcommand{\calN}{\mathcal{N}}
	\newcommand{\tr}{\text{tr}}
	\usepackage{mathtools}
	\DeclarePairedDelimiter\ceil{\lceil}{\rceil}
	\DeclarePairedDelimiter\floor{\lfloor}{\rfloor}
	\DeclareMathOperator{\EX}{\mathbb{E}}% expected value

	
	\newcommand{\note}[1]{{\color{blue} [ #1 ]}}
	\newcommand{\hide}[1]{\underline{\phantom{#1 #1}}}
	
	\newcommand{\R}{\mathbb{R}}
	\newcommand{\Rn}{\mathbb{R}^n}
	\newcommand{\Rnn}{\mathbb{R}^{n \times n}}
	\usepackage{setspace}

\usepackage[T1]{fontenc}
\usepackage{beramono}
\usepackage{listings}
\usepackage[usenames,dvipsnames]{xcolor}

%%
%% Julia definition (c) 2014 Jubobs
%%
\lstdefinelanguage{Julia}%
  {morekeywords={abstract,break,case,catch,const,continue,do,else,elseif,%
      end,export,false,for,function,immutable,import,importall,if,in,%
      macro,module,otherwise,quote,return,switch,true,try,type,typealias,%
      using,while},%
   sensitive=true,%
   alsoother={$},%
   morecomment=[l]\#,%
   morecomment=[n]{\#=}{=\#},%
   morestring=[s]{"}{"},%
   morestring=[m]{'}{'},%
}[keywords,comments,strings]%

\lstset{%
    language         = Julia,
    basicstyle       = \ttfamily,
    keywordstyle     = \bfseries\color{blue},
    stringstyle      = \color{magenta},
    commentstyle     = \color{ForestGreen},
    showstringspaces = false,
}


\newcommand{\handout}[5]{
  \noindent
  \begin{center}
  \framebox{
    \vbox{
      \hbox to 5.78in { {\bf Machine Learning Refined } \hfill #2 }
      \vspace{4mm}
      \hbox to 5.78in { {\Large \hfill #5  \hfill} }
      \vspace{2mm}
      \hbox to 5.78in { {\em #3 \hfill #4} }
    }
  }
  \end{center}
  \vspace*{4mm}
}

\newcommand{\topic}[4]{\handout{#1}{#2}{#3}{Scribe: #4}{Topic: #1}}
\newtheorem{assumption}[theorem]{Assumption}

% 1-inch margins, from fullpage.sty by H.Partl, Version 2, Dec. 15, 1988.
\topmargin 0pt
\advance \topmargin by -\headheight
\advance \topmargin by -\headsep
\textheight 8.9in
\oddsidemargin 0pt
\evensidemargin \oddsidemargin
\marginparwidth 0.5in
\textwidth 6.5in

\parindent 0in
\parskip 1.5ex
%\renewcommand{\baselinestretch}{1.25}

\begin{document}

\topic{Fundamentals of Numerical Optimization --- 3/1 2025}{Spring 2025}{FAUST}{David Zhang}

\section{Overview}

Main discussion points
\begin{itemize}
  \item Gradient Descent 
  \item Newton's Method
\end{itemize}

\subsection{Calculus-defined Optimality: }

\begin{definition} Linear approximation of a function $g$ at a point $v$ is defined as:
  $$h(w) = g(v) + g'(v)(w - v)$$
\end{definition}

where g(v) is the function tangent at v which also contains first derivative information. 

\begin{definition} The quadratic approximation of a function $g$ at a point $v$ is defined as:
  $$h(w) = g(v) + g'(v)(w - v) + \frac{1}{2}(w - v)^Tg''(v)(w - v)$$
\end{definition}

In general we write the linear approximatino as 

$$h(\textbf{w}) = g(\textbf{v}) + \nabla g(\textbf{v})^T(\textbf{w} - \textbf{v})$$
where $\nabla g(\textbf{v})$ is the gradient of $g$ at $\textbf{v}$.
$$\nabla g(\textbf{v}) = \begin{bmatrix}
  \frac{\partial g}{\partial w_1}(\textbf{v}) \\
  \vdots \\
  \frac{\partial g}{\partial w_d}(\textbf{v})
\end{bmatrix}$$

Finding a minimum would be when $\nabla g(\textbf{v}) = \zvec_{N \times 1}$.
These are also called stationary points.

Ideally we would want the function to be convex, so that we can find the global minimum.

In $N$ dimensions, the quadratic funciton in $\textbf{w}$ is defined as:
$$h(\textbf{w}) = \frac{1}{2}\textbf{w}^T\textbf{Qw} + \textbf{r}^T\textbf{w} + d$$.

\subsection{Numerical Methods for Optimization}

$$\textbf{w}^* = \argmin_w g(w)$$

All numerical optimization schemes for minimization work as follows:
\begin{itemize}
  \item Start at some initial point $\textbf{w}^{(0)}$
  \item Update the point iteratively
  \item Stop when some stopping criterion is met
\end{itemize}

\definition{Stopping Condition}
\begin{itemize}
  \item When a pre-specified number of iterations are complete
  \item When the gradient is small enough within an epsilon threshold
\end{itemize}

\subsection{Gradient Descent}

From the first order Taylor Series Approximation centered at $\textbf{w}^{0}$:
$$h(\textbf{w}) \approx g(\textbf{w}^{(0)}) + \nabla g(\textbf{w}^{(0)})^T(\textbf{w} - \textbf{w}^{(0)})$$

Through simple calculus, the steepest descent direction is given as 
$$\textbf{w}^k = \textbf{w}^{k-1}-\nabla_k g(\textbf{w}^{k-1})$$

\subsection{Newton's Method}

Newton's method is a second order optimization method. The idea is to use the second order Taylor Series Approximation centered at $\textbf{w}^{(0)}$:
$$h(\textbf{w}) \approx g(\textbf{w}^{(0)}) + \nabla g(\textbf{w}^{(0)})^T(\textbf{w} - \textbf{w}^{(0)}) + \frac{1}{2}(\textbf{w} - \textbf{w}^{(0)})^T\textbf{Q}(\textbf{w} - \textbf{w}^{(0)})$$

\begin{figure}[h]
\centering
\includegraphics[width=0.5\textwidth]{NewtonsMethod.png}
\caption{Newton's method illustration}
\label{fig:newtons-method}
\end{figure}

Newton's method is often more efficient, but is constrained by the fact that the Hessian matrix must be positive definite.

To do this we can use the first order condition by setting the gradient of h to zero and solving for w. This gives the $N \times N$ system
of linear equations

$$\nabla^2g(\textbf{w}^0)\textbf{w} = \nabla^2g(\textbf{w}^0)\textbf{w}^0 - \nabla(\textbf{w}^0)$$

\begin{algorithm}
\caption{Newton's Method}
\begin{algorithmic}
\State Initialize $\mathbf{w}^{(0)}$
\For{$k=1,2,\ldots$ until convergence}
  \State $\mathbf{w}^{(k)} \leftarrow \mathbf{w}^{(k-1)} 
  - \nabla^2 g\big(\mathbf{w}^{(k-1)}\big)^{-1}\,\nabla g\big(\mathbf{w}^{(k-1)}\big)$
\EndFor
\end{algorithmic}
\end{algorithm}

\begin{definition}
  $\nabla^2 g(\textbf{w}) = \frac{1}{2}(\mQ + \mQ^T)$
\end{definition}

Where the LHS is the Hessian of the matrix.
%\bibliography{mybib}
\bibliographystyle{alpha}

\begin{thebibliography}{77}

\bibitem{fks}
Jeremy Watt\'{o}s, Reza Borhani\'{e} Aggelos K. Katsaggelos,
\emph{Machine Learning Refined: Foundations, Algorithms, and Applications},
Northwestern University.

\end{thebibliography}

\end{document}