\usepackage{amsmath}
\usepackage{amssymb}
\newcommand{\hw}[3]{
	\noindent
	\begin{center}
		\framebox{
			\vbox{
				\hbox to 5.78in { {\bf CSCE 653: Computational Methods for Data Science} \hfill  }
				\vspace{2mm}
				\hbox to 5.78in { {\Large \hfill Homework #1\hfill} }
				\vspace{2mm}
				\hbox to 5.78in { {\it Due date: #2 \hfil Name: #3} }
			}
		}
	\end{center}
	\vspace*{4mm}
}

%Problems, Answers and Solutions
\newcounter{prob}
\setcounter{prob}{-1}
\newcommand{\problem}{\stepcounter{prob}{\noindent\textbf{Problem \theprob.}}\ }
\newcommand{\answer}{\medskip{\color{red} \textbf{Answer to Problem \theprob.}}\ }
\newcommand{\solution}{\medskip{\color{blue} \textbf{Solution to Problem \theprob.}}\ }

%New commands
\def\ds{\displaystyle}
\def\ra{\rightarrow}
%\def\bf{\textbf}
\usepackage{enumerate}
\newcommand{\babc}{\begin{enumerate}[a)]} %Use with \item for abc lists
\newcommand{\eabc}{\end{enumerate}}


% For hyperlinking everything
\usepackage{hyperref}
\hypersetup{
	colorlinks=true, %set true if you want colored links
	linktoc=all,     %set to all if you want both sections and subsections linked
	linkcolor=blue,  %choose some color if you want links to stand out
}

\usepackage{tabularx}
\usepackage{booktabs}
\usepackage[latin1]{inputenc}
\usepackage{amsmath}
\usepackage{mathrsfs}  
\usepackage{amsfonts}
\usepackage{amssymb}
\usepackage{graphicx}
\usepackage{subfig}
\usepackage{caption}
\usepackage{algorithm}
%\usepackage{algcompatible}
%\usepackage{algorithmicx}
\usepackage{algpseudocode}
%\usepackage{enumitem}
\usepackage{color}
\usepackage{titlesec}
\titleformat{\section}{\fontfamily{lmss}\fontsize{14}{15}\bfseries}{\thesection}{1em}{}
\titleformat{\subsection}{\fontfamily{lmss}\fontsize{12}{15}\bfseries}{\thesubsection}{1em}{}

\usepackage{amsthm}

\newtheoremstyle{noit}
{10pt}% <Space above>
{10pt}% <Space below>
{}% <Body font>
{}% <Indent amount>
{\bfseries}% <Theorem head font>
{.}% <Punctuation after theorem head>
{.5em}% <Space after theorem headi>
{}% <Theorem head spec (can be left empty, meaning `normal')>

\newtheoremstyle{example}
{10pt}% <Space above>
{10pt}% <Space below>
{}% <Body font>
{20pt}% <Indent amount>
{\bfseries}% <Theorem head font>
{.}% <Punctuation after theorem head>
{.5em}% <Space after theorem headi>
{}% <Theorem head spec (can be left empty, meaning `normal')>


\newtheoremstyle{indented}{20pt}{20pt}{\addtolength{\leftskip}{2.5em}}{}{\bfseries}{.}{.5em}{}

%\theoremstyle{indented}
\newtheorem{claim}{Claim}
\newtheorem{theorem}{Theorem}
\numberwithin{theorem}{section}
\newtheorem{lemma}[theorem]{Lemma}
\newtheorem{corollary}[theorem]{Corollary}
\newtheorem{observation}{Observation}
%\numberwithin{observation}{section}
%\numberwithin{definition}{section}
\newtheorem{conjecture}{Conjecture}
\newtheorem{Qu}{Question}
\newcommand{\QU}{\begin{Qu}\normalfont}
	
	\newtheorem{definition}{Definition}
	
	\theoremstyle{noit}
	\newtheorem{fact}{Fact}
	
	%\theoremstyle{indented}
	
	
	\theoremstyle{indented}
	\newtheorem{example}{Example}
	
	\theoremstyle{indented}
	%\newtheorem{problem}{Problem}
	
	
	\newcommand{\vs}[1]{\vspace{#1}}
	
	\newcommand{\lecture}[1]{
		\noindent
		\begin{center}
			\framebox{
				\vbox{
					\hbox to 5.78in { {\bf CSCE 653: Computational Methods for Data Science} \hfill  }
					\vspace{2mm}
					\hbox to 5.78in { {\Large \hfill Lecture #1\hfill} }
					\vspace{2mm}
					\hbox to 5.78in { {Texas A\&M University \it  \hfill Lecturer: Nate Veldt} }
				}
			}
		\end{center}
		\vspace*{4mm}
	}
	
	
	\newcommand{\under}[1]{\underline{\hspace{#1}}}
	\setlength{\parindent}{0em}
	
	\newcommand\norm[1]{\left\lVert#1\right\rVert}
	\newcommand\normx[1]{\left\Vert#1\right\Vert}
	
	% Graph terms
	\newcommand{\vol}{\textbf{vol}}
	\newcommand{\cut}{\textbf{cut}}
	%\newcommand{\span}{\textbf{span}}
	
	% Matrices
	\newcommand{\mL}{\textbf{L}}
	\newcommand{\mR}{\textbf{R}}
	\newcommand{\mA}{\textbf{A}}
	\newcommand{\mB}{\textbf{B}}
	\newcommand{\mC}{\textbf{C}}
	\newcommand{\mM}{\textbf{M}}
	\newcommand{\mP}{\textbf{P}}
	\newcommand{\mZ}{\textbf{Z}}
	\newcommand{\mY}{\textbf{Y}}
	\newcommand{\mX}{\textbf{X}}
	\newcommand{\mI}{\textbf{I}}
	\newcommand{\mH}{\textbf{H}}
	\newcommand{\mW}{\textbf{W}}
	\newcommand{\mV}{\textbf{V}}
	\newcommand{\mU}{\textbf{U}}
	\newcommand{\mQ}{\textbf{Q}}
	% vectors
	
	\DeclareMathOperator*{\argmax}{argmax}
	\DeclareMathOperator*{\argmin}{argmin}
	
	\newcommand{\blue}[1]{{\color{blue} #1 }}
	\newcommand{\va}{\textbf{a}}
	\newcommand{\vb}{\textbf{b}}
	\newcommand{\vc}{\textbf{c}}
	\newcommand{\vd}{\textbf{d}}
	\newcommand{\ve}{\textbf{e}}
	\newcommand{\vh}{\textbf{h}}
	\newcommand{\vw}{\textbf{w}}
	\newcommand{\vl}{\textbf{l}}
	\newcommand{\vp}{\textbf{p}}
	\newcommand{\vq}{\textbf{q}}
	\newcommand{\vr}{\textbf{r}}
	\newcommand{\vu}{\textbf{u}}
	\newcommand{\vv}{\textbf{v}}
	\newcommand{\vx}{\textbf{x}}
	\newcommand{\vy}{\textbf{y}}
	\newcommand{\vz}{\textbf{z}}
	\newcommand{\zvec}{\textbf{0}}
	% Other
	\newcommand{\calN}{\mathcal{N}}
	\newcommand{\tr}{\text{tr}}
	\usepackage{mathtools}
	\DeclarePairedDelimiter\ceil{\lceil}{\rceil}
	\DeclarePairedDelimiter\floor{\lfloor}{\rfloor}
	\DeclareMathOperator{\EX}{\mathbb{E}}% expected value

	
	\newcommand{\note}[1]{{\color{blue} [ #1 ]}}
	\newcommand{\hide}[1]{\underline{\phantom{#1 #1}}}
	
	\newcommand{\R}{\mathbb{R}}
	\newcommand{\Rn}{\mathbb{R}^n}
	\newcommand{\Rnn}{\mathbb{R}^{n \times n}}
	\usepackage{setspace}

\usepackage[T1]{fontenc}
\usepackage{beramono}
\usepackage{listings}
\usepackage[usenames,dvipsnames]{xcolor}

%%
%% Julia definition (c) 2014 Jubobs
%%
\lstdefinelanguage{Julia}%
  {morekeywords={abstract,break,case,catch,const,continue,do,else,elseif,%
      end,export,false,for,function,immutable,import,importall,if,in,%
      macro,module,otherwise,quote,return,switch,true,try,type,typealias,%
      using,while},%
   sensitive=true,%
   alsoother={$},%
   morecomment=[l]\#,%
   morecomment=[n]{\#=}{=\#},%
   morestring=[s]{"}{"},%
   morestring=[m]{'}{'},%
}[keywords,comments,strings]%

\lstset{%
    language         = Julia,
    basicstyle       = \ttfamily,
    keywordstyle     = \bfseries\color{blue},
    stringstyle      = \color{magenta},
    commentstyle     = \color{ForestGreen},
    showstringspaces = false,
}
