\documentclass[11pt]{article}
\usepackage{latexsym}
\usepackage{amsmath}
\usepackage{amssymb}
\usepackage{amsthm}
\usepackage{epsfig}
\usepackage{graphicx}

\newcommand{\handout}[5]{
  \noindent
  \begin{center}
  \framebox{
    \vbox{
      \hbox to 5.78in { {\bf CSCE 643 } \hfill #2 }
      \vspace{4mm}
      \hbox to 5.78in { {\Large \hfill #5  \hfill} }
      \vspace{2mm}
      \hbox to 5.78in { {\em #3 \hfill #4} }
    }
  }
  \end{center}
  \vspace*{4mm}
}

\newcommand{\topic}[4]{\handout{#1}{#2}{#3}{Scribe: #4}{Topic #1}}

\newtheorem{theorem}{Theorem}
\newtheorem{corollary}[theorem]{Corollary}
\newtheorem{lemma}[theorem]{Lemma}
\newtheorem{observation}[theorem]{Observation}
\newtheorem{proposition}[theorem]{Proposition}
\newtheorem{definition}[theorem]{Definition}
\newtheorem{claim}[theorem]{Claim}
\newtheorem{fact}[theorem]{Fact}
\newtheorem{assumption}[theorem]{Assumption}

% 1-inch margins, from fullpage.sty by H.Partl, Version 2, Dec. 15, 1988.
\topmargin 0pt
\advance \topmargin by -\headheight
\advance \topmargin by -\headsep
\textheight 8.9in
\oddsidemargin 0pt
\evensidemargin \oddsidemargin
\marginparwidth 0.5in
\textwidth 6.5in

\parindent 0in
\parskip 1.5ex
%\renewcommand{\baselinestretch}{1.25}

\begin{document}

\topic{1 -- The Configuration Space}{Feburary 6th, 2025}{Prof. \ Dylan Shell / Dr. Steven Lavelle}{David Zhang}



\section{Overview}

\begin{definition}
  \textbf{Configuration Space: }The state space for motion planning is a set of possible transformations that could be applied to the robot
\end{definition}

\section{Basic Topological Concepts}

\subsection{Topological Spaces}

\begin{definition}
  \textbf{An open set is } a interval with no boundary points. That must follow this rule (TLDR) 
\end{definition}

\begin{definition}
  \textbf{A topological space} is a set $\mathbb{X}$ together with a collection of open sets $O$ that satisfy the following properties:
  \begin{enumerate}
    \item The union of any number of open sets is an open set.
    \item The intersection of a finite number of open sets is an open set
    \item Both $\mathbb{X}$ and $\emptyset$ are open sets
  \end{enumerate}
\end{definition}

\begin{definition}
 \textbf{Special Points: } A point on the border of an open or closed set
\end{definition}

\subsubsection{Special Point scenarios}
Consider a point $x$ that in the topological space $\mathbb{X}$, and a set $U$ that is a subset of $X$. The following terms
capture the position of point $x$ relative to U.
\begin{enumerate}
  \item If $x$ is in $U$, then $x$ is an interior point of $U$
  \item If $x$ is not in $U$, then $x$ is an exterior point of $U$
  \item If $x$ is neither an interior point nor an exterior point of $U$, then $x$ is a boundary point of $U$
  \item If $x$ is an interior or boundary point, it is a limit point (which the set of all limit points of $U$ 
  is called the closure of $U$)
\end{enumerate}

\em Important Note: The closure is always a closed set, because it contains all the boundary points. That also means
the open set contains none of the boundary points, making it open.

\subsection{The Ball analogy (Example 4.1)}
Think of an open set as a ball in $\mathbb{R}^n$ space, and the points that fill the ball are the interior points
centered at some point $x$.

\begin{enumerate}
  \item All open sets can be represented by a countable union of open balls
  \item Any function constructed from primitives that use the $<$ relation are open.
\end{enumerate}

\subsection{Subspace Topology}
The subspace topology is a topology that have all of its representative open sets be every subset to a larger 
topological space. $U \subseteq \mathbb{X}$

\subsubsection{Blah blah blah}
Here is a subsubsection. You can use these as well.

\subsection{Using Boldface}
Make sure to use lots of boldface.

\paragraph{Question:}
How would you use boldface?

\paragraph{Example:}
This is an example showing how to use boldface to 
help organize your topics.


\paragraph{Some Formatting.}
Here is some formatting that you can use in your notes:
\begin{itemize}
\item {\em Item One} -- This is the first item.
\item {\em Item Two} -- This is the second item.
\item \dots and here are other items.
\end{itemize}

If you need to number things, you can use this style:
\begin{enumerate}
\item {\em Item One} -- Again, this is the first item.
\item {\em Item Two} -- Again, this is the second item.
\item \dots and here are other items.
\end{enumerate}

\paragraph{Bibliography.}
Please give real bibliographical citations for the papers that we
mention in class. See below for how to include a bibliography section.
If you use BibTeX, integrate the .bbl file into your .tex
source. You should reference papers like this: ``The FKS
dictionary originates in a paper by Fredman, Koml\'{o}s and
Szemer\'{e}di \cite{fks}.'' In general, the name of the authors should
appear in text at most once (for the first citation); further
citations look like: ``Our proof follows that of \cite{fks}''.

Take a look at previous topics (TeX files are available) to see the
details. A excellent source for bibliographical citations is
DBLP. Just Google DBLP and an author's name.


%\bibliography{mybib}
\bibliographystyle{alpha}

\begin{thebibliography}{77}

\bibitem{fks}
M. Fredman, J. Koml\'{o}s, E. Szemer\'{e}di,
\emph{Storing a Sparse Table with $O(1)$ Worst Case Access Time},
Journal of the ACM, 31(3):538-544, 1984.

\end{thebibliography}

\end{document}